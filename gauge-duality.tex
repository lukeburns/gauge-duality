\documentclass{article}
\usepackage[margin=1.3in]{geometry}
\usepackage{amsmath}
\usepackage{amsfonts}
\usepackage{multicol}

\title{Gauging Duality Symmetry}
\author{Luke Burns}

\begin{document}
  \maketitle

  \abstract{It is shown that duality symmetry of the generalized Maxwell's equations (with magentic sources) can be gauged. Because the newly introduced gauge field is indistinguishable from an electromagnetic gauge field, it is suggested that the coupling could describe non-linear interactions between electromagnetic fields.}

  \section{Maxwell's Equations}

  An electromagnetic field $F$ is a bivector valued field with vanishing curl. The four Maxwell equations are a consequence of this statement and are given by the single equation\cite{gap}

  \begin{equation}
    \nabla F = J,\label{eq:maxwell}
  \end{equation}

  where 

  \begin{equation}
    F = E + I B
  \end{equation}

  is a frame dependent decomposition of $F$ into a timelike bivector $E$ and spacelike bivector $IB$. In general, 

  \begin{equation}
    J = J_e + I J_b,\label{eq:current}
  \end{equation}

  where $J_e = \nabla \cdot F$ is a vector and $I J_b = \nabla \wedge F$ is a trivector. As suggested by notation, $J_e$ is the electrical current density, and $J_b$ is a hypothetical magnetic current density.

  The restriction that $F$ has vanishing curl

  \begin{equation}
    \nabla \wedge F = 0\label{eq:constraint}
  \end{equation}

  is the same as requiring that $J_b = 0$. The ``generalized'' Maxwell's equations, without the constraint that $J_b=0$, are solely determined by the fact that $F$ is a bivector. In other words, \emph{any} bivector field contains a decomposition into electric and magnetic parts that satisfy the generalized Maxwell's equations (Equation \ref{eq:maxwell}).

  \section{Generalized Potentials}

  By allowing $F$ to be an arbitrary bivector field, it can be defined in terms of a potential $M$ as

  \begin{equation}
    F = \langle \nabla M \rangle_2, \label{eq:potential-eq}
  \end{equation}

  where the brackets are an indication to only take the bivector part of $\nabla M$.

  $\nabla M$ only contains even grade terms if $M$ contains odd grade terms. Assuming $M$ contains no even terms (which would have no contribution to $F$), it is of the form

  \begin{equation}
    M = M_e + I M_b, \label{eq:potential}
  \end{equation}

   where $M_e$ is a vector and $I M_b$ is a trivector such that

  \begin{equation}
    F = F_e + F_b = \nabla \wedge M_e + \nabla \cdot I M_b,
  \end{equation}

  which ensures that the electric and magnetic source terms 

  \begin{equation}
    J_e = \nabla \cdot (\nabla \wedge M_e) \text{ and } J_b = \nabla \wedge (\nabla \cdot I M_b)
  \end{equation} 

  are sourced by the corresponding vector potentials $M_e$ and $M_b$. Note that $M$ is the usual electromagnetic potential when $M_b = 0$. 

  $F$ is invariant under

  \begin{equation}
    M \mapsto M - \nabla \phi,
  \end{equation}

  where

  \begin{equation}
    \phi = \phi_e + I \phi_b,
  \end{equation}

  for scalar and pseudoscalar fields $\phi_e$ and $I \phi_b$. A similar construction can be found in \cite{dressel}. \cite{cagc} showed that any multivector field $F$ has an anti-derivative if it is integrable, so we can be assured that the $M$ actually exists. Furthermore, we are guaranteed that the antiderivative $M$, before making any gauge transformations, satisfies

  \begin{equation}
    F = \nabla M = \nabla \wedge M_e + \nabla \cdot I M_b
  \end{equation}

  which automatically comes with the constraints

  \begin{equation}
    \nabla \cdot M_e = \nabla \wedge I M_b = 0,
  \end{equation}

  which is just the Lorenz gauge for both electric and magnetic potentials. $M$ is given explicitly by a generalized Cauchy's integral formula \cite{cagc}

  \begin{equation}
  M(x) = \int G(x, x') F(x') |d^{m}x'| - \oint G(x, x') n^{-1} d^{m-1}x' M(x')\right),
\end{equation}

where $n$ is the unit normal and $G$ is the green's function for $\nabla$.

  \section{Duality Transformations}

  A duality rotation of $F$

  \begin{equation}
    F \mapsto F e^{I \phi} = \cos(\theta) E - \sin(\theta) B + I(B \cos(\theta) + E \sin(\theta)),\label{eq:duality}
  \end{equation}

  mixes up electric and magnetic fields. If this is a global, constant transformation, then Equation \ref{eq:maxwell} is invariant given an identical transformation of $J$

  \begin{equation}
    J \mapsto J e^{I \phi},
  \end{equation}

  which mixes up electric and magnetic sources. Because of this mixing of electric and magnetic sources, the duality transformation does not preserve the physical content of the theory, and so cannot be considered a true symmetry of the equation (unless $J=0$). On the other hand, duality transformations are a \emph{formal} symmetry of Equation \ref{eq:maxwell}. And seeing that the presence of magnetic sources already challenges the physical integrity of $J$, it is arguably less wrong to call it a symmetry in the general case where $J$ includes magnetic sources.

  For the case $J=0$, duality is a true global symmetry of Equation \ref{eq:maxwell}, so one might ask whether this symmetry can be promoted to a local symmetry. Interestingly, while Equation \ref{eq:maxwell} with $J=0$ is invariant under duality transformations, the traditional (sourceless) Lagrangian

  \begin{equation}
    \mathcal{L}_e = \frac{1}{2} \langle F_e^2 \rangle.\label{eq:elagrangian}
  \end{equation}

  is \emph{not} (brackets denote the scalar part). This has caused some confusion regarding the possibility of gauging duality symmetry. 

  Additionally, \cite{saa} reviewed attempts of \cite{malik}, \cite{bunster}, and \cite{deser} to gauge this symmetry. \cite{bunster} and \cite{deser} concluded that duality symmetry could not be gauged, in contradiction to the result of \cite{malik}, who arrived at the equation

  \begin{equation}
    \nabla F = I A F,\label{eq:malik}
  \end{equation}

  where $J_A \equiv IAF$ behaves like a current, including magnetic sources.

  \cite{saa} concluded that the discrepancy between these results was due to the fact that \cite{bunster} and \cite{deser} required preservation of gauge invariance, whereas \cite{malik} failed to consider this (i.e. that the gauging process breaks gauge invariance of the original field).  

  However, the potential used in the argument of \cite{saa} was a complex 3-vector potential, which is guaranteed to satisfy the vacuum equations. Failure of gauge invariance is to be expected, seeing that Equation \ref{eq:malik} involves the electric and magnetic current density $J_A$.

  The confusion is cleared up by introduction of a full multivector valued potential $M$ --- a potential for both electric and magnetic sources. By a similar method, \cite{dressel} and \cite{vasconcellos} constructed dual symmetric Lagrangians.

  Ultimately, the no-go conclusions reached by \cite{bunster} and \cite{deser} were due to the fact that the resulting gauge field introduces both electric \emph{and} magnetic sources into the Equation \ref{eq:maxwell}. This is the reason for loss of gauge invariance and related to the lack of local symmetry at the level of the action. 

  \cite{saa}, \cite{malik}, \cite{bunster}, and \cite{deser} only considered the vacuum equations. A potential for a vacuum field does not admit sources, so the vacuum gauge field description cannot continue to be used after gauging. Hence, loss of gauge invariance. 

  The usual (sourceless) Lagrangian given by Equation \ref{eq:elagrangian} fails to be symmetric because it's missing its ``magnetic'' counterpart. A dual symmetric Lagrangian (for the vaccum equations) can be written\cite{dressel}

  \begin{equation}
    \mathcal{L} = \frac{1}{2}\langle F_e^2 + F_b^2 \rangle.
  \end{equation}

  \cite{tiwari} constructed a dual symmetric Lagrangian by a different method and arrived at the same equations as \cite{malik} and generalized the result for the case where $J \not= 0$.

  Note that the lack of symmetry of Equation \ref{eq:elagrangian} is a distinct issue from the loss of gauge invariance. To maintain gauge invariance, the potential must be promoted to a full multivector as in Equation \ref{eq:potential}.

  Here, we will present a derivation of the results of \cite{tiwari} at the field level, directly from Equation \ref{eq:maxwell}, without assuming Equation \ref{eq:constraint}. If we allow duality transformations to vary, letting $\phi = \phi(x)$, then we will pick up an extra term in the derivative

  \begin{equation}
    \nabla (F e^{I \phi}) = \nabla F e^{I \phi} - I \nabla \phi F e^{I\phi},
  \end{equation}

  where the negative sign is due to the fact that the pseudoscalar $I$ anticommutes with $\nabla$.

  We can construct a covariant derviative $D$ with an added gauge field that transforms in such a way to absorb the contributions of this local symmetry to the derivative. Namely, $D$ must transform so that

  \begin{equation}
    D' F' = D' (F e^{I \phi}) = D F e^{I \phi}. \label{eq:covariant}
  \end{equation}

  To accomplish this, we'll tack on a term to the derivative

  \begin{equation}
    D = \nabla - I A
  \end{equation}

  that transforms as

  \begin{equation}
    A \mapsto A' = A - \nabla \phi\label{eq:gaugefreedom}
  \end{equation}

  in tandem with duality transformations. The quantity $A$ must be the same grade as $\nabla \phi$, so it is a vector whose curvature $F_A = \nabla \wedge A$ satisfies Maxwell's equations without magnetic sources.\footnote{Actually, it could technically include a trivector term and preserve the form of the equation. It may be interesting to try gauging duality plus ``scaling'' transformations of the form $e^{\phi_b + I \phi_e}$.} Equation \ref{eq:gaugefreedom} is equivalent to

  \begin{equation}
    D \mapsto D' = D + I \nabla \phi,
  \end{equation}

  which ensures Equation \ref{eq:covariant} holds, yielding the equation

  \begin{equation}
    \nabla F - I A F = J.\label{eq:gauged}
  \end{equation}

  Equation \ref{eq:gauged} is the same as the result of \cite{tiwari}, as well as \cite{malik} for the case $J=0$.

  $IA$ itself is a pseudo-vector, and was correctly identified to transform as such by \cite{malik} and \cite{tiwari}. Due to its transformation properties, \cite{malik}, \cite{naik}, and \cite{pmn} proposed that it gave rise to long-range spin interactions mediated by an axial vector boson (they called it an ``axial photon''). \cite{tiwari}  drew connections to axion electrodynamics. \cite{vasconcellos} constructed identified a relation to electroweak gauge fields.

  Identifying the gauge field

  \begin{equation}
    IA = \nabla I a,
  \end{equation}

  where $Ia$ is a pseudoscalar ``axion'' field, and requiring $J_b = 0$ and $A \cdot F = 0$ results in the equations of axion electrodynamics:

  \begin{equation}
    \nabla F = J_e + (\nabla I a) \cdot F.
  \end{equation}

  Interestingly, in this case, $A = -\nabla a$ is a gradient of a scalar field, which implies that the curvature $F_A = \nabla \wedge A = 0$. Normally, this field would be considered to be absent of physical content. 

  The fact that $IA$ is a trivector is expected when compared with the U(1) gauge theory in quantum mechanics. Gauge theory usually involves operators $i \partial_\mu$ instead of just $\partial_\mu$. In this case, the pseudoscalar $I$ takes on the dual roles of the unit imaginary $i$, since $I^2 = -1$ and of the duality (Hodge) map between vectors and pseudo-vectors.

  The replacement 

  \begin{equation}
    D \mapsto - I D = -I \nabla - A = \gamma^\mu (I \partial_\mu - A_\mu)
  \end{equation}

  places the gauging procedure performed here and in quantum mechanics on comparable footing, and we can see that the gauge field here is no different from the U(1) gauge field of electromagnetism in Dirac theory.

  This raises the question of whether the field $A$ can be identified directly with an electromagnetic field. From this perspective, the gauge field would not give rise to a new ``axial photon,'' as suggested by \cite{malik}, \cite{naik}, and \cite{pmn}. The gauge boson would simply be a photon. In which case, Equation \ref{eq:gauged} would describe a pair of coupled electromagnetic fields $F$ and $F_A = \nabla \wedge A$ interacting non-linearly. Physically comprehensible interactions that do not involve magnetic sources would be described by the equation

  \begin{equation}
    \nabla F - I A F = J_e,
  \end{equation}

  where the pseudovector part $\nabla \wedge F - I A \cdot F = J_b = 0$ vanishes. However, $J_e$ would necessarily pick up magnetic source terms under a gauge transformation, so the above gauge field must be fixed.

  \begin{thebibliography}{9} 
    \bibitem{gap} 
      C. Doran and A. Lasenby.
      \emph{Geometric Algebra for Physicists}. Cambridge University Press (2003).

    \bibitem{cagc}
      D. Hestenes.
      \emph{Clifford algebra to geometric calculus}.
      D. Reidel Publishing Company (1984).

    \bibitem{dressel} 
      J. Dressel, K. Y. Bliokh, and F. Norib.
      \emph{Spacetime algebra as a powerful tool for electromagnetism}.
      Physics Reports 589 (2015).

    \bibitem{saa}
      Alberto Saa.
      \emph{Local electromagnetic duality and gauge invariance}.
      Classical and Quantum Gravity 28 (2011).

    \bibitem{malik}
      R. P. Malik and T. Pradhan 
      \emph{Local Duality Invariance of Maxwell's Equations}.
      Z. Phys. C - Particles and Fields 28 (1985).

    \bibitem{bunster}
      C.~Bunster and M.~Henneaux,
        \emph{Can (Electric-Magnetic) Duality Be Gauged?}.
        Phys. Rev. D 83, 045031 (2011).
        arXiv:1011.5889.

    \bibitem{deser}
        S. Deser.
        \emph{No local Maxwell duality invariance}.
        Classical and Quantum Gravity
        28 (2011).
        arXiv:1012.5109.

    \bibitem{pmn}
      T. Pradhan, R. P. Malik, and P. C. Naik.
      \emph{The fifth interaction: universal long range force between spins}.
      Pramana 24 (1985). 

    \bibitem{tiwari}
      S.C. Tiwari.
      \emph{Axion electrodynamics in the duality perspective}
      Modern Physics Letters A.
      Vol. 30, No. 40 (2015).

    \bibitem{vasconcellos}
      C. A. Z. Vasconcellos and D. Hadjimichef.
      \emph{CP Violation in Dual Dark Matter}.
      Astron. Nachr. 335 (2014).
      arXiv:1404.0409.

    \bibitem{naik}
      P. C. Naik and T. Pradhan.
      \emph{Long-range interactions between spins}.
      Journal of Physics A: Mathematical and General 14 (1981).

  \end{thebibliography}

\end{document}